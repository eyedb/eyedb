\documentclass[dvips,10pt]{book}
%\documentclass[dvips,10pt]{report}

\usepackage{latexsym}
%\usepackage{amstex}
%\usepackage{amssymb}
%\usepackage{amsfonts}
\usepackage{amstext}
\usepackage{amsbsy}
\usepackage{amscd}
\usepackage{makeidx}
\usepackage{graphicx}
%\usepackage{calc}
\usepackage{longtable}

% -- to reduce the margin pages --
\topmargin 2pt
\headsep 20pt
\headheight 2pt

%\textwidth 440pt
%\textheight 690pt
%\oddsidemargin 0pt
%\evensidemargin 0pt

% 19/01/06: EV
\textwidth 490pt
\textheight 700pt
\oddsidemargin -20pt
\evensidemargin -20pt

% --------------------------------

\makeindex

\newcommand{\eyedb}{\textsc{EyeDB} }
\newcommand{\eyedbX}{\textsc{EyeDB}}
\newcommand{\sysra}{SYSRA}
\newcommand{\indexentry}[2]{\item #1 #2}
\newcommand{\chap}[1]{\chapter{#1} \index{#1}}
\newcommand{\sect}[1]{\section{#1} \index{#1}}
\newcommand{\subsect}[1]{\subsection{#1} \index{#1}}
\newcommand{\subsubsect}[1]{\subsubsection{#1} \index{#1}}
%% WARNING \Box does not work with latex2html
%%\newcommand{\iii}{\item[{$\Box$}]}
\newcommand{\rrr}{\item[{$\Box$}]}
\newcommand{\iii}{\item}
\newcommand{\jjj}{\item[-]}
\newcommand{\bi}{\begin{itemize}}
\newcommand{\ei}{\end{itemize}}
\newcommand{\be}{\begin{enumerate}}
\newcommand{\ee}{\end{enumerate}}
\newcommand{\kkk}{\item[ ]}
\newcommand{\sh}[1]{\bf #1}
\newcommand{\ttt}{$\tilde{ }$}
\newcommand{\ttv}[1]{\texttt{#1}}
\newcommand{\ttu}[1]{{\verbsize\texttt{#1}}}
\newcommand{\verbsizex}{\small}
\newcommand{\verbsizey}{\fontsize{8}{\baselineskip}\selectfont}
\newcommand{\verbsize}{\normalsize}
\newcommand{\bugreport}{bug-report@eyedb.org }
\newcommand{\idt}{\mbox{ } - }
\newcommand{\ident}{{\bf \texttt{<}identifier\texttt{>}}}
\newcommand{\grindent}{xxxxxxxxxxxxxxx \= : \kill}
\newcommand{\arr}{\texttt{-$>$}}
\newcommand{\ul}[1]{\underline{#1}}
\newcommand{\uul}[1]{\textit{\texttt{#1}}}
\newcommand{\ixx}{\makebox[3mm]{ }}
\newcommand{\ixy}{\makebox[10mm]{ }}
\newcommand{\ixz}{\\\ixy {- }}

\renewcommand{\normalsize}{\fontsize{9}{11pt}\selectfont}

%\renewcommand{\normalsize}{\fontsize{8}{10pt}\selectfont}

%% The following commands comes from corbaref.tex

\newcommand{\bt}{\begin{tabular}{|p{7cm}|p{7cm}|}}
\newcommand{\et}{\hline \end{tabular} \\ {\vspace{0.6cm}} \\ }

\newcommand{\mt}[1]{\hline \multicolumn{2}{|c|}{\centerline{{\bf interface #1}
   \emph{mapped from} {\bf class #1}}}}

\newcommand{\Desc}[1]{\hline \multicolumn{2}{|l|}{\emph{#1}}}

\newcommand{\mTT}[2]{\hline \multicolumn{2}{|c|}
   {\centerline{{\bf interface #1} \emph{extends} {\bf #2}
   \emph{mapped from} {\bf #1}}}}


\newcommand{\mT}[2]{\hline \multicolumn{2}{|c|}{\centerline{{\bf #1}
\emph{mapped from} {\bf #1}}}}

\newcommand{\att}{\hline \hline {\bf Attribute} & {\bf Description}}
\newcommand{\mtt}{\hline \hline {\bf Method} & {\bf Description}}
\newcommand{\mf}{\emph{mapped from }}

\newcommand{\satt}[1]{\hline \multicolumn{2}{|c|}
   {\centerline{\emph{attributes inherited from} {\bf #1}}}}

\newcommand{\smtt}[1]{\hline \multicolumn{2}{|c|}{
   \centerline{\emph{methods inherited from} {\bf #1}}}}

\newcommand{\sets}[1]{sets the #1 \emph{value} in the Object \emph{o} at the \emph{ind}}

\newcommand{\Gets}[1]{gets the #1 value in the Object \emph{o} at the \emph{ind}}

%\setcounter{secnumdepth}{-1}
%\addtocounter{secnumdepth}{2}

\renewcommand{\thesection}{\arabic{section}}
\renewcommand{\thesubsection}{\arabic{section}.\arabic{subsection}}
\renewcommand{\thechapter}{}

\begin{document}
\normalsize


\input{version}
\newcommand{\mantitle}{\textsc{Object Definition Language}}

\thispagestyle{empty}

\mbox{ }
\\
\vspace{3 cm}
\\
%{\bf{\Huge E{\LARGE YE}DB} \hspace{0.cm} {\LARGE Programming Manual}}
{\bf{\Huge E{\LARGE YE}DB} \hspace{0.cm} {\Huge \mantitle}}
\\
%\newcommand{\rulewidth}{12.2cm}
%\newcommand{\rulewidth}{15.4cm}
\newcommand{\rulewidth}{17.2cm}
\rule{\rulewidth}{1.6mm}
\begin{flushright}
{\large Version \eyedbversion}
\end{flushright}
\mbox{ }
\\
\vspace{11 cm}
\\
\begin{flushright}
{\large December 2007}
\end{flushright}

\newpage

\thispagestyle{empty}

\mbox{ }
\\
\vspace{1cm}
\\
Copyright {\copyright} 1994-2008 SYSRA
\\
\vspace{1cm}
\\
Published by \sysra
\\
30, avenue G\'en\'eral Leclerc
\\
91330 Yerres - France
\\
\\
home page: http://www.eyedb.org
%\\
%\\
%bug report: \bugreport



\tableofcontents

\chapter*{The Object Definition Language}

The \eyedb Object Definition Language (ODL) is a specification language
to define the specifications of object types based on the ODMG ODL (but not compliant).\\
ODL \footnote{ODL is used for shortness to denote \eyedb ODL} 
is not intended to be a full programming language. It is
a definition language for object specifications. Database management
systems traditionally provide facilities that support
data definition (using a Data Definition Language (DDL)). The DDL allows
users to define their data types and interfaces while the
Data Manipulation Language (DML) allows to create, delete, read update
instances of those data types.\\
ODL is a DDL for objects types.
If defines the characteristics of types, including their properties
and operations.
ODL defines only the signatures of operations defined in C++ and does not
address definitions of the methods that implements those operations.
Operations defined in OQL can be defined in the ODL.
\\
ODL is intended to define object types that can be implemented
in a variety of programming languages. Therefore, ODL is not tied
to the syntax of a particular programming language.\\\\
\eyedb ODL differs from ODMG ODL from several points:
\bi
\item ODMG ODL defines class attributes, relationships, method signatures and keys.
It supports nested classes, typedef constructs, constant definitions and
exception hints.
\item \eyedb ODL defines class attributes, relationships, method signatures,
attribute constraints (notnull, unique, collection cardinality), index
specifications and trigger declarations. It does not support
nested classes, typedef constructs, constant definitions and
exception hints.
\item in \eyedb ODL, any type instance can be both a literal or
an object. In ODMG ODL, this property is tied to the type: all
basic types and user defined \emph{struct} are literal while \emph{interface}s
and \emph{class}es are objects.
In \eyedb ODL, any type instance can be an object, even the basic types.
\item at last, \eyedb ODL allows to specify whether a method is executed by the server or by the client,
and whether it is a class or instance method.
\ei

\sect{The Language Specifications}

The basic concept of the \eyedb object model is the class which, as in any
traditional object model, model a set
of objects of similar properties (attributes) and behaviors (methods).
The attributes can be basic types, user types, references, arrays, collections.
The methods can be defined in C++ or in OQL (Object Query Language).
\\
\\
ODL allows one to specify classes, attributes, methods, triggers, constraints,
enumerate types, indexes and implementation hints.
\\
\\
We are going to introduced in details all the features of ODL.
\subsect{Comments}
The ODL comments are like in C++:
\bi
\item mono-line comments: any characters following \texttt{//} and until
the end of the line are comments
\item multi-line comments: any characters (including newlines) between \texttt{/*} and \texttt{*/}
are comments
\ei
For instance:
\vspace{-2mm}
\begin{verbatim}
// this is a simple line comments

/* this is
   a multi line
   comments */
\end{verbatim}
\subsect{Basic types}
The basic types are as follows:\\\\
\begin{tabular}{|p{2cm}|p{5cm}|}
\hline
byte & 1-byte integer \\
\hline
char & 1-byte character \\
\hline
short & 2-byte integer \\
\hline
int & 4-byte integer \\
\hline
long & 8-byte integer \\
\hline
double & 8-byte floating point \\
\hline
oid & 8-byte internal object identifier \\
\hline
enum & 4-byte integer \\
\hline
\end{tabular}
\\
\\
For instance:
\vspace{-0.2cm}
\begin{verbatim}
class C {
  attribute byte b;
  attribute char c;
  attribute short s;
  attribute int i;
  attribute long l;
  attribute double d;
  attribute oid o;
};
\end{verbatim}
Notes :
\be
\vspace{-0.2cm}
\item The key word \texttt{attribute} is optional:
\vspace{-0.2cm}
\begin{verbatim}
class C {
  byte b;
  char c;
  // ...
};
\end{verbatim}
\vspace{-0.2cm}
is correct.
\item The grammar does not allow one to gather several attributes on
the same line declaration:
\vspace{-0.2cm}
\begin{verbatim}
class C {
  attribute char c1, c2, c3; // NOT correct
  // ...
};
\end{verbatim}
\ee
\subsect{Enum types}
An enumerate type is denoted by a set of integers mapped to symbols
like in C++.
The syntax is similar to the C++ syntax, for instance:
\vspace{-0.2cm}
\begin{verbatim}
enum E1 {
  A, // A == 0
  B, // B == 1
  C  // C == 2
};

enum E2 {
  D = 3,   // D == 3
  E,       // E == 4
  F = 100, // F == 100
  G,       // G == 101
  H        // H == 102
};

class C {
  attribute int i;
  E1 e1;
  E2 e2;
};
\end{verbatim}
\subsect{Array types}
The object model supports multi-dimensional fixed or variable size arrays
of any type. For instance:
\vspace{-0.2cm}
\begin{verbatim}
class C {
  attribute byte b_a[4];      // fixed length mono-dimensional array
  attribute char str[];       // variable size mono-dimensional array
  attribute int i_a[3][4][8]; // multi-dimensional fixed size array
  attribute long l_a[][4][8]; // multi-dimensional variable size array
};
\end{verbatim}
%\vspace{-0.2cm}
One particular interesting array type is the array of characters, which
can be denoted as \texttt{string} as follows:
\vspace{-0.2cm}
\begin{verbatim}
class C {
  attribute string s;       // <=> char s[] (unlimited size string)
  attribute string<32> bs;  // <=> char bs[32] (bounded string)
};
\end{verbatim}
%\vspace{-0.2cm}
Note that in a multi-dimensional array, only the extreme left dimension can be
variable:
\vspace{-0.2cm}
\begin{verbatim}
class C {
  attribute long l_a1[][4][8]; // correct
  attribute long l_a2[4][][8]; // NOT correct
  attribute long l_a2[4][8][]; // NOT correct
};
\end{verbatim}
\subsect{Literal and object types}
Remember that each object in a set of \eyedb databases has an unique identifier called OID.
\\
A literal attribute is an attribute fully included in the class and has
no OID, while an object attribute denotes the reference
to another object with an object identifier.
A reference attribute is denoted by a \texttt{*} or a \texttt{\&} symbol.
For instance:
\vspace{-0.2cm}
\begin{verbatim}
class C1 {
  attribute int i;
};

class C {
  attribute C1  l_c1;  // literal attribute included in C
  attribute C1 *o_c1;  // object attribute referenced by C (or &oc1_1)
};
\end{verbatim}
Let \texttt{c} an instance of the class \texttt{C}.
\bi
\item \texttt{c} includes a literal of type \texttt{C1} through the attribute
\texttt{l\_c1}
\item \texttt{c} can reference an object of type \texttt{C1} through the attributes 
\texttt{o\_c1}
\item if \texttt{c} is removed from the database, the attribute \texttt{l\_c1}
is removed at the same time, but the object denoted by \texttt{o\_c1} is
not removed
\ei
\begin{tabular}{|p{16.8cm}|}
\hline
\vspace{0.1mm}
Do not confuse the \texttt{*} ODL meaning and the \texttt{*} C/C++ meaning:
in C/C++, the \texttt{*} type modifier denotes an address to an area of the
indicated type instances: it is a pointer to an address. This pointer
can be incremented and decremented to change its location in the area.
\\
In ODL, the \texttt{*} denotes a reference to one and only one object, it is why
the \texttt{\&} token is also accepted, although the meaning of this token
is a little bit different in C++.
\\
\\
So, in ODL the construct \texttt{C1 **oc1} makes no sense, in the same
manner that the construct \texttt{C1 \&\&oc1} makes no sense in C++.\\
\hline
\end{tabular}\\
\\
One can have arrays of literal or object as follows:
\vspace{-0.2cm}
\begin{verbatim}
class C {
  attribute C1  l_c1_1[2];
  attribute C1  l_c1_2[];
  attribute C1  l_c1_3[][10][20];

  attribute C1 *o_c1_1[4];
  attribute C1 *o_c1_2[];
  attribute C1 *o_c1_3[][4][5];
};
\end{verbatim}

\subsect{Collection types}
The \eyedb object model support three types of collections, set, bag and array. A fourth type, list, will be implemented in a further version:
\bi
\item a set an unordered collection of elements of the same type
not allowing duplicate elements
\item a bag a unordered collection of elements of the same type
allowing duplicate elements
\item an array an ordered collection of elements of the same type
allowing duplicate elements
\item a list (\emph{non yet implemented}) is an ordered collections of elements of
the same type allowing duplicate elements
and where element insertion and removal is
efficiently implemented
\ei
An element may be of any type, literal or object and
a collection attribute may be a literal or an object, and one can have
arrays of collection, for instance:
\vspace{-0.2cm}
\begin{verbatim}
class C {
  attribute set<int>  i_lset;    // literal set of int
  attribute set<C1>   l_c1_lset; // literal set of C1 literals
  attribute set<C1 *> o_c1_lset; // literal set of C1 objects

  attribute set<int>  *i_oset;    // object set of int
  attribute set<C1>   *l_c1_oset; // object set of C1 literals
  attribute set<C1 *> *o_c1_oset; // object set of C1 objects

  attribute bag<C1 *> o_c1_lbag;  // literal bag of C1 objects

  attribute array<C1 *> o_c1_larr;   // literal array of C1 objects
  attribute bag<C1 *>   o_c1_lbag[]; // array of literal bag of C1 objects

  // multi-dimensional array of literal bag of set of array of C1 objects
  attribute bag<set<array<set<C1 *> > > > x[2][3][4]; 
};
\end{verbatim}
The differences between an array collection (i.e.
\texttt{array<}\emph{type}\texttt{>} and an attribute array (i.e. 
\emph{type} \texttt{[]}) are:
\bi
\item a collection array may exists independently from any class as a
an attribute array exists only within a class
\item the implementation is very different:
\bi
\item one can have a big collection array (thousand or millions of elements)
without loss of performance (if the collection is well parameterised, see
below)). Big attribute array are unefficient
\item collection array can have ``holes'' without loss of performance, for instance an element at index 1 and another one at index 1000000 and nothing between. An attribute array with holes are unefficient as they are stored
consecutively
\item a collection array is heavier than an attribute array, and so is not
recommended for little size
\ei
\ei

\subsect{Inheritance}
The object model support single inheritance using the keyword \texttt{extends}:
\vspace{-0.2cm}
\begin{verbatim}
class C1 {
  attribute string c1;
};

class C2 extends C1 {
  attribute string c2;
};

class C3 extends C2 {
  attribute string c3;
};
\end{verbatim}
As in usual object conception, an object of class \texttt{C2} includes the two
attributes \texttt{c1} and \texttt{c2} and an object of class \texttt{C3}
includes the three attributes \texttt{c1}, \texttt{c2} and \texttt{c3}.
\\
\\
In the following construct:
%\vspace{-0.2cm}
\begin{verbatim}
class C4 {
  attribute C1 *oc1;
  attribute C2 *oc2;
  attribute C3 *oc3;

  attribute C1 lc1;
  attribute C2 lc2;
  attribute C3 lc3;
};
\end{verbatim}
The attribute \texttt{oc1} may be of type \texttt{C1}, \texttt{C2} or
\texttt{C3}.\\
The attribute \texttt{oc2} may be of type \texttt{C2} or \texttt{C3}.\\
The attribute \texttt{oc3} may be of type \texttt{C3} only.\\
The attribute \texttt{lc1} is of type \texttt{C1}.\\
The attribute \texttt{lc2} is of type \texttt{C2}.\\
The attribute \texttt{lc3} is of type \texttt{C3}.


\subsect{Constraints}
The object model supports currently two declarative constraints: notnull
and unique. The cardinality constraint on collection is partially
implemented and is not currently supported.
Non declarative constraints are defined using triggers (see below).
\\
\\
Note that:
\bi
\item unique constraint cannot be defined on several attributes and
\item unique constraint on an attribute needs an index. The index is not
automatically created, it must be defined in the ODL (see below)
or outside using the \texttt{idxcreate} tool.
\ei
For instance:
\vspace{-0.2cm}
\begin{verbatim}
class C {
  attribute string s1;
  attribute string s2;
  attribute string s3;

  constraint<notnull> on s1;

  constraint<notnull> on s2;
  constraint<unique> on s2;

  constraint<unique> on s3;
};
\end{verbatim}
The attribute \texttt{s1} must not be null.\\
The attribute \texttt{s2} must not be null and is unique in the collection
of C objects.\\
The attribute \texttt{s3} is unique in the collection
of C objects.
\subsubsect{Constraint and inheritance propagation}
By default, constraints are propagated to subclasses, let \texttt{C2}
a subclass of \texttt{C}:
\vspace{-0.2cm}
\begin{verbatim}
class C2 extends C {
  attribute string c2;
};
\end{verbatim}
When one creates an \texttt{C2} object, the attributes \texttt{s1} and
\texttt{s2} must not be null and the attributes \texttt{s2} and
\texttt{s3} must be unique.\\
\\
Important note: the unique constraint applies separately on each class
(\texttt{C} and \texttt{C2}) and not on the set of inheritance class tree.
This means that one can have a \texttt{C} object with a given value
for \texttt{s2} and a \texttt{C2} object with the same value for \texttt{s2}.
\\
This is not the expected default behavior and will
be parameterised in a next version.
\\
\\
If you do no want to propagate automatically a constraint to the subclasses,
you need to use the construct \texttt{propagate = off} as follows:
\vspace{-0.2cm}
\begin{verbatim}
class C {
  attribute string s1;
  attribute string s2;
  attribute string s3;

  constraint<notnull, propagate = off> on s1;

  constraint<notnull> on s2;
  constraint<unique, propagate = off> on s2;

  constraint<unique> on s3;
};

class C2 extends C {
  attribute string c2;
};
\end{verbatim}
The notnull constraint on \texttt{C::s1} and the unique constraint
on \texttt{C1::s2} will not be propagated to \texttt{C2}, but the
notnull constraint on \texttt{C::s1} and the unique constraint on
\texttt{C::s3} will be propagated to \texttt{C2}.

\subsubsect{Constraint on attribute of literal composite type}
One can define constraints on attributes of literal composite type
attribute, for instance:
\vspace{-0.2cm}
\begin{verbatim}
class C1 {
  attribute string s1;
  attribute int i1;
};

class C {
  attribute C1 c1;
 
  constraint<notnull> on c1.s1;
  constraint<unique> on c1.i1;
};
\end{verbatim}

\subsect{Referential integrity}
The \eyedb object model support one-to-one, one-to-many and many-to-many
relationships. 
\\
\\
A relationship between a class \texttt{A} and a class \texttt{B} is
materialized by attributes in the two classes of the following
types according to the cardinality of the relationship:
\bi
\item one-to-one : \texttt{A} contains an attribute of type \texttt{B *} and
\texttt{A} contains an attribute  of type \texttt{B *}
\item one-to-many : \texttt{A} contains an attribute of type \emph{collection}\texttt{<B *>} (\emph{collection} is a set or a bag) and
\texttt{A} contains an attribute  of type \texttt{B *}
\item many-to-many : \texttt{A} contains an attribute of type \emph{collection}\texttt{<B *>} and
\texttt{A} contains an attribute  of type \emph{collection}\texttt{<B *>}
\ei
For instance for a one-to-one relationship:
\vspace{-0.2cm}
\begin{verbatim}
class A {
  attribute string sa;
  attribute B *b;
};

class B {
  attribute string sb;
  attribute A *a;
};
\end{verbatim}
In the previous case, \eyedb maintains only partially the referential
integrity: for instance, one cannot create an object \texttt{A}
with an attribute \texttt{b} which refers an non-existent \texttt{B}
object. But, if the referenced \texttt{B} object is removed, the attribute
\texttt{b} will still referenced the removed object.
\\
\\
\eyedb can maintain the referential integrity by indicating the
\texttt{inverse directive} in the ODL as follows:
\vspace{-0.2cm}
\begin{verbatim}
class A {
  attribute string sa;
  relationship B *b inverse B::b; // or inverse b
};

class B {
  attribute string sb;
  relationship A *a inverse A::b; // or inverse a
};
\end{verbatim}
Note \texttt{attribute} has been replaced by \texttt{relationship} in
this case: this is mandatory.
\\
In this case, if the \texttt{B} object referenced by a \texttt{A} object
through \texttt{b} is removed, \texttt{b} is set to the null value.
\\
A one-to-many relationship:
\vspace{-0.2cm}
\begin{verbatim}
class A {
  attribute string sa;
  relationship set<B *> b_set inverse a;
};

class B {
  attribute string sb;
  relationship A *a inverse b_set;;
};
\end{verbatim}
and a many-to-many relationship:
\vspace{-0.2cm}
\begin{verbatim}
class A {
  attribute string sa;
  relationship set<B *> b_set inverse a_set;
};

class B {
  attribute string sb;
  relationship set<A *> a_set inverse b_set;;
};
\end{verbatim}

\subsect{Methods}
In ODL, one can declare the signature of C++ and OQL methods
and one can defined the body of OQL methods.
By default, a method is executed on the server side.
\\
A method argument can be any basic type, reference on a composite type or
mono-dimensional array of basic or composite type. An argument
can be \texttt{in}, \texttt{out} or \texttt{inout}.
Argument may be named or unnamed (only type is given), for instance:
\vspace{-0.2cm}
\begin{verbatim}
class C1 {
  attribute string c1;
};

class C2 {
  attribute string c2;
  int perform(in int size, in string str, out double, in C1 &, inout C2 &);
};
\end{verbatim}
Note that the \texttt{\&} symbol may be replaced by the \texttt{*} symbol
or no symbol as anyhow only a persistent object (not a litteral) may
be passed to a method call.
\\
\\
The \texttt{C::perform} method must be defined in C++ but may be called
from OQL or a C++ client. To define a C++ method, refer to the
document \emph{C++ Binding}.
\\
\\
Methods can be overloaded (same name but different signatures), for instance:
\vspace{-0.2cm}
\begin{verbatim}
class C2 {
  attribute string c2;
  int perform(in int size, in string str, out double, in C1 &, inout C2 &);
  int perform(in double, out string mystr);
};
\end{verbatim}
One can define OQL methods in ODL. In this case, the name of the
arguments must be given:
\vspace{-0.2cm}
\begin{verbatim}
class C2 {
  attribute string c2;
  int append(in string s)
  %oql{
    this.s2 += s;
    return strlen(this.s);
  %};
};
\end{verbatim}
The OQL \texttt{this} variable denotes the calling instance.
\\
\\
A method can be an instance method (the default) or a class method (equivalent
to C++ or Java static methods). To defined a class method, there are
two constructs, using the keyword \texttt{static} or \texttt{classmethod}:
\vspace{-0.2cm}
\begin{verbatim}
class C {
  static int perform1(in string); // or
  classmethod int perform2(in string);

  instmethod int perform3(in string); // <=> int perform3(in string)
};
\end{verbatim}
If you want to execute a method on the client side, you must
use the keyword \texttt{client} as follows:
\vspace{-0.2cm}
\begin{verbatim}
class C {
  instmethod<client> int perform1(in string);
  classmethod<client> int perform2(in string);

  instmethod<server> int perform3(in string);  // <=> int perform3(...)
  classmethod<server> int perform4(in string); // <=> classmethod perform3(...)
};
\end{verbatim}

\subsect{Triggers}
Triggers are server methods which are executed when a particular
event occurs on an object: before or after creation, update, load or delete.
\\
Like methods, a trigger can be written in C++ or in OQL. On the other
hand a trigger has no argument but has a name;
\vspace{-0.2cm}
\begin{verbatim}
class C {
  attribute string s;

  // C++ triggers
  trigger<create_before> c_b();
  trigger<create_after> c_a();

  trigger<update_before> u_b();
  trigger<update_after> u_a();

  trigger<load_before> l_a();
  trigger<load_after> l_b();

  trigger<remove_before> r_b();
  trigger<remove_after> r_a();

  trigger<create_before> c_b2(); // one can have several create_before triggers

  // OQL trigger
  trigger<create_before> l_a2()
  %oql{
    if (strlen(this.s) > 100)
      throw "invalid length";
  %};
};
\end{verbatim}


\subsect{Indexes}
Indexes can be either defined in ODL or with the tool \texttt{eyedbidxadmin}.
To define indexes on attributes:
\begin{verbatim}
class C {
  attribute string s;
  attribute int i;

  index on s;
  index on i;
};
\end{verbatim}
Note that we cannot define one index on several attributes.
\\
\subsubsect{Index and inheritance propagation}
As constraints, indexes may be or not propagated to subclasses.
The behavior is the same as for constraints: indexes are propagated
by default to subclasses:
\begin{verbatim}
class C {
  attribute string s;
  attribute int i;

  index on s;
  index on i;
};

class C2 extends C {
  attribute long l;
  
};
\end{verbatim}
Indexes are created for \texttt{C::s}, \texttt{C::i}, \texttt{C2::s}
and \texttt{C2::i}.
\\
\\
Note that the index on \texttt{C::s} (resp. \texttt{C::i} is different from the
index on \texttt{C2::s} (resp. \texttt{C2::i}).
\\
To forbid propagation:
\begin{verbatim}
class C {
  attribute string s;
  attribute int i;

  index<propagate=off> on s;
  index<propagate=off> on i;
};

class C2 extends C {
  attribute long l;
  
};
\end{verbatim}
Indexes are created only for only \texttt{C::s} and \texttt{C::i}.
\subsubsect{Index on attribute of literal composite type}
One can create indexes on an attribute of a literal composite type, for
instance:
\begin{verbatim}
class C1 {
  attribute int i;
  attribute double d;
};

class C {
  attribute string s;
  C1 c1; // literal composite type

  index on s;
  index on c1.i;
  index on c1.d;
};
\end{verbatim}
\subsubsect{Index specifications}
By default, an index on a number attribute (char, short, int, long and double)
is implemented as a BTree, while an index on either a string or a bounded string is
implemented as a Hash index.
\\
The differences between BTree and Hash are as follows:
\bi
\item BTree indexes allows one to retrieve in an efficient way
entries with values greater or lesser than a given value. Hash indexes
does not allows this in a efficient way.
\item On big volume of data, BTree indexes are more efficient 
with the default parameters than Hash indexes with the default parameters
\item On the other hand, for exact match search, a Hash index
with good parameters is more efficient than a BTree index
\item Usually, Hash index creation is at least four times faster
than creating a BTree index
\ei
The ODL index specification allows one to change the default index
type and parameters of a given target attribute.
To set the type of an index on a given attribute, one uses:
\begin{verbatim}
class C {
  attribute string<32> s;
  attribute int i;

  index<type = btree> on s; // default is hash: change to btree
  index<type = hash> on i;  // default is btree: change to hash
};
\end{verbatim}
Important note: one cannot create a BTree index on a non bounded string.
One can set implementation parameters for indexes as follows:
\begin{verbatim}
class C {
  attribute string<32> s;
  attribute int i;

  index<type = btree, hints = "degree = 64;"> on s;
  index<type = hash, hints = "key_count = 4096; initial_size = 4096;
        extend_coef = 1; size_max = 4096;"> on i;
};
\end{verbatim}

%\subsect{Advances features}
%  collection implementation
%  class implementation ??
%  name alias
%  quoted sequences

\sect{The eyedbodl tool}
The eyedbodl tool can be used to:
\bi
\item update a database from an ODL file:
\bi
\item create a schema
\item add methods, triggers, constraints, indexes to classes
\item remove methods, triggers, constraints, indexes to classes
\item add classes
\item remove classes
\item add attributes
\item rename attributes
\item remove attributes
\item remove classes
\ei
\item generate C++ stubs from an ODL file or a database containing
a schema
\item generate Java stubs from an ODL file or a database containing
a schema
\item generate ODL from a database containing a schema
\item display the differences between an ODL file and a database
schema
\item check an ODL file syntax
\ei
For instance, let \texttt{schema.odl} an ODL file and \texttt{dbtest}
a database.
\subsect{Updating a schema}
To update a database from an ODL file:
\vspace{-2mm}
\begin{verbatim}
eyedbodl -u -d dbtest schema.odl
\end{verbatim}
or:
\vspace{-2mm}
\begin{verbatim}
eyedbodl --update --database=dbtest schema.odl
\end{verbatim}
Important notes:
\bi
\item All classes defined in the ODL file will be added to the
existing schema in the database
\item The classes in the database and not in the ODL file will not
be removed from the database
\item To remove a class from a database, one must use the
\texttt{--rmcls=}\emph{class} option. Because of class dependancies,
the removal of a class can fail because one needs to remove
other classes, for instance collection classes of the class one want
to remove.
\\
In this case, one must delete classes in the good order.
\item To remove a entire schema from a database, on must use
the \texttt{--rmsch} option
\item The methods, triggers, constraint and indexes
in the ODL file and not in the database will be added to the database
\item The methods, triggers, constraint and indexes in the databae
and not in the ODL file will {\bf not} be removed from the database
unless the \texttt{-rmv-undex-attrcomp=yes} option is given
\item The common indexes (on same attributes) in the ODL file and in the
database with a different implementation will not be updated unless
the \texttt{--update-index=yes} is given
\item A class \texttt{C} defined in the ODL and in the database with
different attributes will have the following behavior:
\bi
\item an attribute \texttt{a} in a class \texttt{C} of the ODL, not
in the class \texttt{C} of the database will be automatically added to the
class \texttt{C} in the database
\item an attribute \texttt{a} in a class \texttt{C} in the database, not
in the class \texttt{C} in ODL will be automatically {\bf removed} from
the class \texttt{C} in the database: {\bf This operation is not undoable}
\item an attribute \texttt{a} in a class \texttt{C} in the database and
in the class \texttt{C} in ODL with different types will lead
to an update failure
\ei
\ei
\subsect{Generates C++ code}
To generate the C++ API from an ODL file:
\vspace{-2mm}
\begin{verbatim}
eyedbodl --gencode=C++ schema.odl
\end{verbatim}
To generate the C++ API from a database:
\vspace{-2mm}
\begin{verbatim}
eyedbodl --gencode=C++ --package=schema -d dbtest
\end{verbatim}
For a given \emph{package}.odl ODL file, the generated files are as follows:
\bi
\item \emph{package}.h, \emph{package}.cc: the generated C++ API to be used in
a client program
\item template\_\emph{package}.cc: an example of a client program using the generated
API
\item Makefile.\emph{package}: an example of Makefile to compile \emph{package}.cc and
template\_\emph{package}.cc: make -f Makefile.\emph{package} will compile
and link the generated API and template files
\item \emph{package}stubsfe.cc, \emph{package}stubsbe.cc: stubs for client and server methods
\item \emph{package}mthfe-skel.cc, \emph{package}mthbe-skel.cc: skeletons for
client and server methods
\ei

%Quoted sequence

\subsect{Generates Java code}
To generate the Java API from an ODL file:
\vspace{-2mm}
\begin{verbatim}
eyedbodl --gencode=Java schema.odl
\end{verbatim}
To generate the Java API from a database:
\vspace{-2mm}
\begin{verbatim}
eyedbodl --gencode=Java --package=schema -d dbtest
\end{verbatim}
For a given \emph{package}.odl ODL file, 
the \emph{package} directory contains a Java file for each class
defined in the ODL file plus a Java file for each collection template
class used as an attribute in classes of the ODL file.
\subsect{Generates ODL}
To generate the ODL from a database:
\vspace{-2mm}
\begin{verbatim}
eyedbodl --gencode=ODL -d dbtest # generates on the standard ouput
eyedbodl --gencode=ODL -d dbtest -o schema.odl
\end{verbatim}

\subsect{Display schema differences}
To display the difference between a schema in an ODL file and a database
schema:
\vspace{-2mm}
\begin{verbatim}
eyedbodl --diff schema.odl -d dbtest
\end{verbatim}

\subsect{Checking the syntax of an ODL file}
To check the syntax of an ODL file:
\vspace{-2mm}
\begin{verbatim}
eyedbodl --checkfile schema.odl
\end{verbatim}

Beside these major options, eyedbodl has a lot of extra options as
described when running eyedbodl with the --help option.
\newpage
\sect{Annexes}
\subsect{A simple example}
Here is a simple example that can be found
in examples/C++Binding/schema-oriented/share/schema.odl:
\begin{verbatim}
enum CivilState {
  Lady = 0x10,
  Sir  = 0x20,
  Miss = 0x40
};

class Address {
  attribute string street;
  attribute string<32> town;
  attribute string country;

  index on street;
};

class Person {
  attribute string name;
  attribute int age;
  attribute Address addr;
  attribute Address other_addrs[];
  attribute CivilState cstate;
  attribute Person * spouse inverse Person::spouse;
  attribute set<Car *> cars inverse owner;
  attribute array<Person *> children;
  int change_address(in string street, in string town,
                     out string oldstreet, out string oldtown);
  index on  name;
};

class Car {
  attribute string brand;
  attribute int num;
  Person *owner inverse cars;
};

class Employee extends Person {
  attribute long salary;
};
\end{verbatim}
\newpage
\subsect{A more complex example}
Here is a more complex example used for the management of biological
databases:
\begin{verbatim}
enum StatusType {
  running = 0,
  done = 1
};

class File {
  attribute string path;
  attribute string name;
  attribute string desc;
  attribute set<Import_ctx *> imported_in inverse Import_ctx::file;

  constraint<notnull, propagate=on> on name;
};

class Import_ctx {
  attribute File * file inverse File::imported_in;
  attribute Import * import inverse Import::contexts;
  attribute StatusType status;
  attribute string comment;
  attribute int32 count;
  attribute int32 elapsed;
  attribute float average;
  attribute string start_date;
  attribute string last_update;

  constraint<notnull, propagate=on> on file;
  constraint<notnull, propagate=on> on import;
};

class Import {
  attribute Db * related_db inverse Db::imports;
  attribute string database_name;
  attribute string cvs_tag;
  attribute set<Import_ctx *> contexts inverse Import_ctx::import;
  attribute string comment;
  attribute bool deletable;

  instance_method <client, called_from=OQL> time_interval getElapsed()

  constraint<unique, propagate=on> on database_name;
  constraint<notnull, propagate=on> on database_name;
  constraint<unique, propagate=on> on cvs_tag;
  constraint<notnull, propagate=on> on related_db;

  index< propagate=on> on database_name;
  index< propagate=on> on cvs_tag;
};

class Db {
  attribute string name;
  attribute string title;
  attribute int32 version;
  attribute set<Import *> imports inverse Import::related_db;
  attribute array<File *> files;
  attribute set<Db *> divisions;
  attribute Import * official;

  instance_method <client, called_from=OQL> string [] get_db_names();

  constraint<unique, propagate=on> on name;
  constraint<notnull, propagate=on> on name;

  index< propagate=on> on name;
};
\end{verbatim}
\newpage
\subsect{The eyedbodl usage}
The usage of the eyedbodl is as follows:
\begin{verbatim}
eyedbodl --gencode=C++ [--package=<package>] [--output-dir=<dirname>] [--output-file-prefix=<prefix>]
         [--schema-name=<schname>] [--namespace=<namespace>] [--class-prefix=<prefix>]
         [--db-class-prefix=<dbprefix>] [--attr-style=implicit|explicit] [--dynamic-attr]
         [--gen-class-stubs] [--class-enums=yes|no] [--c-suffix=<suffix>] [--h-suffix=<suffix>]
         [--export] [--down-casting=yes|no] [--gencode-error-policy=status|exception] [--attr-cache=yes|no]
         [--rootclass=<rootclass>] [--no-rootclass] [--cpp=<cpp>] [--cpp-flags=<flags>]
         [--no-cpp] <odlfile>|-|-d <dbname>|--database=<dbname> [<openflags>]

eyedbodl --gencode=Java --package=<package> [--output-dir=<dirname>] [--output-file-prefix=<prefix>]
         [--schema-name=<schname>] [--class-prefix=<prefix>] [--db-class-prefix=<dbprefix>]
         [--attr-style=implicit|explicit] [--dynamic-attr] [--down-casting=yes|no]
         [--gencode-error-policy=status|exception] [--cpp=<cpp>] [--cpp-flags=<flags>]
         [--no-cpp] <odlfile>|-|-d <dbname>|--database=<dbname> [<openflags>]

eyedbodl --gencode=ODL -d <dbname>|--database=<dbname> [--system-class]
         [-o <odlfile>] [<openflags>]
eyedbodl --diff -d <dbname>|--database=<dbname> [--system-class] [<openflags>] [--cpp=<cpp>]
         [--cpp-flags=<flags>] [--no-cpp] <odlfile>|-

eyedbodl -u|-update -d <dbname>|--database=<dbname> [--db-class-prefix=<dbprefix>] [<openflags>]
         [--schema-name=<schname>] [--rmv-undef-attrcomp=yes|no] [--update-index=yes|no]
         [--cpp=<cpp>] [--cpp-flags=<flags>] [--no-cpp] [--rmcls={<class>}] [--rmsch] [<odlfile>|-]

eyedbodl --checkfile <odlfile>|-

eyedbodl --help

One must specify one and only one of the following major options:
--gencode=C++                 Generates C++ code
--gencode=Java                Generates Java code
--gencode=ODL                 Generates ODL
--update|-u                   Updates schema in database <dbname>
--diff                        Displays the differences between a database schema and an odl file
--checkfile                   Check input ODL file
--help                        Displays the current information

The following options must be added to the --gencode=C++ or Java option:
<odlfile>|-|-d <dbname>|--database=<dbname> Input ODL file (or - for standard input) or the database name

The following options can be added to the --gencode=C++ or Java option:
--package=<package>           Package name
--output-dir=<dirname>        Output directory for generated files
--output-file-prefix=<prefix> Ouput file prefix (default is the package name)
--class-prefix=<prefix>       Prefix to be put at the begining of each runtime class
--db-class-prefix=<prefix>    Prefix to be put at the begining of each database class
--attr-style=implicit         Attribute methods have the attribute name
--attr-style=explicit         Attribute methods have the attribute name prefixed by get/set (default)
--schema-name=<schname>       Schema name (default is <package>)
--export                      Export class instances in the .h file
--dynamic-attr                Uses a dynamic fetch for attributes in the get and set methods
--down-casting=yes            Generates the down casting methods (the default)
--down-casting=no             Does not generate the down casting methods
--attr-cache=yes              Use a second level cache for attribute value
--attr-cache=no               Does not use a second level cache for attribute value (the default)

For the --gencode=C++ option only
--namespace=<namespace>       Define classes with the namespace <namespace>
--c-suffix=<suffix>           Use <suffix> as the C file suffix
--h-suffix=<suffix>           Use <suffix> as the H file suffix
--gen-class-stubs             Generates a file class_stubs.h for each class
--class-enums=yes             Generates enums within a class
--class-enums=no              Do not generate enums within a class (default)
--gencode-error-policy=status Status oriented error policy (the default)
--gencode-error-policy=exception Exception oriented error policy
--rootclass=<rootclass>       Use <rootclass> name for the root class instead of the package name
--no-rootclass                Does not use any root class

The following options can be added to the --gencode=ODL option:
--system-class                Generates system class ODL

The following option must be added to the --update|-u option:
-d <dbname>|--database=<dbname> Database for which operation is performed

The following options can be added to the --update|-u option:
<odlfile>|-                   Input ODL file or '-' (standard input)
--schema-name=<schname>       Schema name (default is package)
--db-class-prefix=<prefix>    Prefix to be put at the begining of each database class
--rmv-undef-attrcomp=yes|no   Removes (yes) or not (no) the undefined attribute components
                              (constraint, index and implementation). Default is no
--update-index=yes|no         Updates (yes) or not (no) the index with a different
                              implementation in the DB. Default is no
--rmcls={<class>}             Removes the given class list
--rmsch                       Removes the entire schema

The following options must be added to the --diff option:
-d <dbname>|--database=<dbname> Database for which the schema difference is performed
<odlfile>                       The input ODL file for which the schema difference is performed

The following options can be added to the --diff option:
--system-class                Performs difference on system classes also

The following option must be added to the --checkfile option:
<odlfile>|-                   Input ODL file or '-' (standard input)

The following options can be added when an <odlfile> is set:
--cpp=<cpp>                   Uses <cpp> preprocessor instead of the default one
--cpp-flags=<cpp-flags>       Adds <cpp-flags> to the preprocessing command
--no-cpp                      Does not use any preprocessor

Common Options:
  -U <user>|@, --user=<user>|@       User name
  -P [<passwd>], --passwd[=<passwd>] Password
  --host=<host>                      eyedbd host
  --port=<port>                      eyedbd port
  --inet                             Use the tcp_port variable if port is not set
  --dbm=<dbmfile>                    EYEDBDBM database file
  --conf=<conffile>                  Configuration file
  --logdev=<logfile>                 Output log file
  --logmask=<mask>                   Output log mask
  --logdate=on|off                   Control date display in output log
  --logtimer=on|off                  Control timer display in output log
  --logpid=on|off                    Control pid display in output log
  --logprog=on|off                   Control progname display in output log
  --error-policy=<value>             Control error policy: status|exception|abort|stop|echo
  --trans-def-mag=<magorder>         Default transaction magnitude order
  --arch                             Display the client architecture
  -v, --version                      Display the version
  --help-eyedb-options               Display this message
\end{verbatim}
\end{document}





\newpage
\subsect{A very complex example}
The following schema has been used in a biological projects:
\verbsize
\begin{verbatim}
class Date {
  attribute int32 release;
  attribute int32 version;
  attribute date d;
};

class TopicComment {
  attribute string name;
  attribute string comment;
};

class DbObject {
  attribute string name;
  attribute ostring other_names[]; 
  attribute array<EntryReference *> accession_numbers;
  attribute string description;
  attribute Date creation_date;
  attribute Date last_update_date;
  attribute Date last_annotation_date;
  attribute bag<CrossReference *> cross_references;
  attribute array<Citation *> references;
  attribute set<KeyWord *> keywords;
  attribute array<Author *> authors;
  attribute string comment;
  attribute TopicComment topicComments[];

  index< propagate=on, type=hash> on name;
  index< propagate=on, type=hash> on other_names.s; 
  index< propagate=on> on accession_numbers;
  index< propagate=on> on cross_references;
  index< propagate=on> on keywords;
  index< propagate=on> on authors;
  index< propagate=on> on references;
  index< propagate=on> on creation_date.d.julian;
  index< propagate=on> on DbObject.last_annotation_date.d.julian;
  index< propagate=on> on DbObject.last_update_date.d.julian;
  index< propagate=on> on topicComments.name;
  index< propagate=on> on topicComments.comment;
};

class DbReference {
  attribute string<32> id;
  attribute int32 version;
  attribute string sid;
  attribute Db * db;

  index< propagate=on, type=hash> on id;
  index< propagate=on, type=btree> on db;
  instance_method <client, called_from=OQL> oid get_entry_oid()
  %oql{
    return null;
   %}
};

class Db {
  attribute string name;
  attribute string title;
  attribute string code;
  attribute int32 version;
  attribute set<Division *> divisions;
  attribute set<DbReference *> accession_numbers;

  index< propagate=on> on name;
  index< propagate=on> on code;
  index< propagate=on> on divisions;
  index< propagate=on> on accession_numbers;
};

class Division extends Db {
  attribute Db * db;
};

class EntryReference extends DbReference {
  attribute DbObject * db_object;

  index< propagate=on> on db_object;
  instance_method <client, called_from=OQL> oid get_entry_oid()
  %oql{
    return this.db_object;
  %}
};

class CrossReference extends DbReference {
  attribute string status;
  attribute DbObject * db_object;

  index< propagate=on> on db_object;
  instance_method <client, called_from=OQL> oid get_remote_oid()
  %oql{
    dbname := this.db.name;
    id := this.id;
    return getDbObject(dbname, id);
  %}

  instance_method <client, called_from=OQL> oid get_entry_oid()
  %oql{
    return this.get_remote_oid();
   %}
};

class KeyWord {
  attribute string kvalue;
  attribute bag<DbObject *> db_objects;

  index< propagate=on> on kvalue;
  index< propagate=on> on db_objects;
};

class Author {
  attribute string name;
  attribute set<DbObject *> db_objects;

  index< propagate=on> on name;
};

class Reference extends DbObject {
  attribute string title;
  attribute string location;

  index< propagate=on> on title;
  index< propagate=on> on location;
};

class Citation extends Reference {
  attribute int32 num;
  attribute Location * positions;
};

class Location {
};

class CommentLocation extends Location {
  attribute string comment;
};

class BioObject extends DbObject {
  attribute set<Organism *> organisms;
  attribute set<Organelle *> organelles;

  index< propagate=on> on organisms;
  index< propagate=on, type=btree> on organelles;
};

class Organism {
  attribute string genus;
  attribute ostring names[];
  attribute Taxon * taxonomy;
  attribute bag<BioObject *> bioobjects;

  index< propagate=on> on genus;
  index< propagate=on> on bioobjects;
  index< propagate=on> on taxonomy;
  index< propagate=on> on names.s;
};

class Taxon {
  attribute string name;
  attribute Taxon * parent;
  attribute set<Taxon *> children;
  attribute set<Organism *> organisms;
  attribute Taxonomy * taxonomy;

  index< propagate=on> on name;
  index< propagate=on> on children;
  index< propagate=on> on organisms;
};

class Taxonomy {
  attribute string title;
  attribute set<Taxon *> taxons;
};

class Organelle {
  attribute string name;
  attribute Organism * organism;
  attribute bag<BioObject *> bioobjects;

  index< propagate=on> on name;
  index< propagate=on> on organism;
  index< propagate=on> on bioobjects;
};

enum SequenceStrand {
  unknown_orientation = 0,
  direct = 1,
  complementary = 2,
  both = 3,
  both_reverse = 4
};

enum CompositionType {
  join = 0,
  order = 1,
  group = 2,
  oneof = 3
};

enum MoleculeType {
  mol_unknown_type = 0,
  mol_dna = 1,
  mol_rna = 2,
  mol_aa = 3,
  mol_na = 4,
  mol_other_type = 255
};

enum MoleculeTopology {
  mol_unknown_topology = 0,
  mol_linear = 1,
  mol_circular = 2,
  mol_tandem = 3,
  mol_other_topology = 255
};

class SequenceLocation extends Location {
  attribute SequenceStrand strand;
};

class RangeSequenceLocation extends SequenceLocation {
  attribute int32 from_beyond;
  attribute int32 from_location;
  attribute int32 to_location;
  attribute int32 to_beyond;
};

class SiteSequenceLocation extends SequenceLocation {
  attribute int32 after_base;
};

class SingleBaseFromRangeSequenceLocation extends SequenceLocation {
  attribute RangeSequenceLocation  location;
};

class SingleBaseFromSetSequenceLocation extends SequenceLocation {
  attribute int32 bases[];
};

class RemoteSequenceLocation extends SequenceLocation {
  attribute CrossReference * sequence;
};

class RemoteLocatedSequenceLocation extends RemoteSequenceLocation {
  attribute SequenceLocation * location;
};

class RemoteLabeledSequenceLocation extends RemoteSequenceLocation {
  attribute string label;
};

class ReferenceSequenceLocation extends SequenceLocation {
  attribute string base_string;
};

class CompositeSequenceLocation extends SequenceLocation {
  attribute array<SequenceLocation *> locations;
  attribute CompositionType composition_type;
};

class FeatureQualifier {
  attribute string<21> name;
  attribute string qvalue;
};

class Feature {
  attribute string<17> key;
  attribute SequenceLocation * location;
  attribute string description;
  attribute FeatureQualifier qualifiers[];
};

class SequenceObject extends BioObject {
  attribute MoleculeType molecule;
  attribute Feature feature_table[];
  attribute int32 length;
  attribute string sequence;
  attribute BioLibrary * library;
  attribute Clone * clone;
  attribute MoleculeTopology topology;

  index< propagate=on, type=hash> on molecule;
  index< propagate=on, type=btree> on length;
  index< propagate=on> on library;
  index< propagate=on> on clone;
  index< propagate=on> on feature_table.key;
  index< propagate=on> on feature_table.description;
  index< propagate=on> on feature_table.qualifiers.name;
  index< propagate=on> on feature_table.qualifiers.qvalue;
};

class SequenceAlignment {
  attribute DbReference * sequence_ref;
  attribute int32 start;
  attribute int32 end;
  attribute int32 length;
  attribute float score;
  attribute string aligned_portion;
  attribute Experiment * computation;

  index< propagate=on, type=btree> on length;
  index< propagate=on> on sequence_ref;
  index< propagate=on> on computation;
};

class NucleicSequenceComposition {
  attribute int32 a;
  attribute int32 c;
  attribute int32 g;
  attribute int32 t;
  attribute int32 other;
};

class NucleicSequence extends SequenceObject {
  attribute NucleicSequenceComposition  composition;
};

class Gene extends BioObject {
  attribute ostring names[];
  attribute set<ProteicSequence *> proteinSequences;

  index< propagate=on> on proteinSequences;
};

class ProteicSequence extends SequenceObject {
  attribute int32 molweight;
  attribute set<Gene *> genes;
  attribute set<CrossReference *> domains;
  index< propagate=on> on genes;
};

class ProteinFamily extends BioObject {
  attribute SequenceAlignment alignment[];
  attribute int32 number_of_domains;

  index< propagate=on, type=btree> on number_of_domains;
};

class BioLibrary {
  attribute string name;
  attribute string id;
  index< propagate=on> on name;
  index< propagate=on> on id;
};

class EST extends NucleicSequence {
  attribute string<10> end;

  index< propagate=on> on end;
};

class Compound {
  attribute string name;

  index< propagate=on> on name;
};

class ReactionSubstrate {
  attribute Compound * substrate;
  attribute int32 stoechiometry;
};

class Reaction {
  attribute ReactionSubstrate reactants[];
  attribute ReactionSubstrate products[];
  attribute string description;
};

class Cofactor {
  attribute set<Compound *> compounds;
  index< propagate=on> on compounds;
};

class Enzyme extends DbObject {
  attribute ostring alternate_names[];
  attribute Reaction  reaction;
  attribute Cofactor cofactors[];
};

class CytoElement extends BioObject {
};

class CytoBand extends CytoElement {
};

class Chromosome extends CytoElement {
};

enum MapElementType {
  is_framework = 0,
  is_multipoint = 1,
  is_pairwise = 2,
  is_placement = 3
};

enum MarkerType {
  is_unclassified = 0,
  is_genetic = 1,
  is_EST = 2,
  is_full_cDNA = 3,
  is_CDS = 4,
  is_alternative_genetic = 5,
  is_mRNA = 6,
  is_CpGisland = 7,
  is_other = 8,
  is_SNP = 9
};

class MapObject extends BioObject {
  attribute set<MapElement *> on_maps;
};

class MapElement {
  attribute MapObject * mappable;
  attribute GenomeMap * map;
  attribute MapElementType position_type;
  attribute float dist_top;
  attribute float dist_bottom;
  attribute float lod_top;
  attribute float lod_bottom;

  index< propagate=on, type=hash> on position_type;
  index< propagate=on> on mappable;
  constraint<notnull, propagate=on> on mappable;
  index< propagate=on> on map;
  constraint<notnull, propagate=on> on map;
};

class GenomeMap extends BioObject {
  attribute Chromosome * chromosome;
  attribute float lod_score;
  attribute array<MapElement *> elements;
  index< propagate=on, type=btree> on lod_score;
  index< propagate=on> on chromosome;
};

class Clone extends MapObject {
};

class Experiment extends DbObject {
};

class Panel extends BioObject {
  attribute int32 nb_elements;
  attribute Clone clones[];
};

class Marker extends MapObject {
  attribute MarkerType types[];
};

class Assay {
  attribute BioObject * bioobject;
  attribute Experiment * protocol;

  index< propagate=on> on bioobject;
  index< propagate=on> on protocol;
};

class STS extends Marker {
  attribute string CA_primer;
  attribute string GT_primer;
  attribute int32 PCR_length;
};

class RhMap extends GenomeMap {
  attribute Panel * panel;
  index< propagate=on> on panel;
};

class MapInterval {
  attribute MapElement * mapobj1;
  attribute MapElement * mapobj2;

  index< propagate=on> on mapobj1;
  index< propagate=on> on mapobj2;
};

enum DataClass {
  standard = 0,
  preliminary = 1

};

class EmblEntry extends NucleicSequence {
  attribute DataClass data_class;
};

class SwissprotEntry extends ProteicSequence {
  attribute DataClass data_class;
  attribute string crc;
};

enum DomainType {
  is_pattern = 0,
  is_matrix = 1,
  is_rule = 2

};

class DbScan {
  attribute Db * db;
  attribute int32 nb_sequences;
  attribute int32 total_sequences;
  attribute int32 total_hits;
  attribute int32 positive_sequences;
  attribute int32 positive_hits;
  attribute int32 unknown_sequences;
  attribute int32 unknown_hits;
  attribute int32 false_positive_sequences;
  attribute int32 false_positive_hits;
  attribute int32 false_negative_sequences;
  attribute int32 false_negative_hits;
  attribute int32 partial_sequences;
  attribute int32 partial_hits;
  attribute set<CrossReference *> hits;
};

class PrositeEntry extends ProteinFamily {
  attribute set<Documentation *> documentation;
  attribute DomainDefinition * definition;
  attribute DbScan  proteins;
  attribute set<Taxon *> taxonomic_range;
  attribute int32 max_nb_of_repeats;
  attribute Feature sites[];
  attribute bool skip_flag;
  attribute set<CrossReference *> structure_3D;
  attribute DomainType domain_type;

  index< propagate=on, type=hash> on domain_type;
  index< propagate=on, type=btree> on max_nb_of_repeats;
  index< propagate=on> on documentation;
  index< propagate=on, type=btree> on proteins.total_sequences;
  index< propagate=on, type=btree> on proteins.total_hits;
  index< propagate=on, type=btree> on proteins.positive_sequences;
  index< propagate=on, type=btree> on proteins.positive_hits;
  index< propagate=on, type=btree> on proteins.unknown_sequences;
  index< propagate=on, type=btree> on proteins.unknown_hits;
  index< propagate=on, type=btree> on proteins.false_positive_sequences;
  index< propagate=on, type=btree> on proteins.false_positive_hits;
  index< propagate=on, type=btree> on proteins.false_negative_sequences;
  index< propagate=on, type=btree> on proteins.false_negative_hits;
  index< propagate=on, type=btree> on proteins.partial_sequences;
  index< propagate=on, type=btree> on proteins.partial_hits;
  index< propagate=on> on proteins.db;
};

class Documentation extends DbObject {
  attribute string title;
  attribute string text;
  attribute set<DbObject *> entries;
  index< propagate=on> on title;
  index< propagate=on> on entries;
};

class DomainDefinition {
};

class DomainRule extends DomainDefinition {
  attribute string rule;
  index< propagate=on> on rule;
};

class DomainProfile extends DomainRule {
  attribute string matrix;
  attribute string matrix_type;
  attribute string scaling_db;
  attribute Feature  region;
  index< propagate=on> on matrix;
};

class DomainPattern extends DomainRule {
  attribute string pattern;
  index< propagate=on> on pattern;
};

class ProtomatResults {
  attribute string<32> code;
  attribute int32 width;
  attribute int32 sequences;
  attribute float proba;
  attribute int32 score;
  attribute int32 strength;
};

class BlocksEntry extends ProteinFamily {
  attribute int32 distance_from_previous_block1;
  attribute int32 distance_from_previous_block2;
  attribute Experiment * computation;
  attribute ProtomatResults  protomat_results;

  index< propagate=on, type=btree> on distance_from_previous_block1;
  index< propagate=on, type=btree> on distance_from_previous_block2;
  index< propagate=on> on computation;
  index< propagate=on, type=btree> on protomat_results.code;
  index< propagate=on, type=btree> on protomat_results.width;
  index< propagate=on, type=btree> on protomat_results.sequences;
  index< propagate=on, type=btree> on protomat_results.score;
  index< propagate=on, type=btree> on protomat_results.strength;
};

class FrequentString {
  attribute string name;
  attribute int32 occurences;
};

class FamilyObtention {
  attribute string description;
  attribute CrossReference * protein_family;
};

class ProdomEntry extends ProteinFamily {
  attribute FrequentString frequent_protein_name[];
  attribute int32 alignment_length;
  attribute int32 diameter;
  attribute int32 radius_of_gyration;
  attribute SequenceAlignment  sequence_closest_to_consensus;
  attribute FamilyObtention  obtention;
  attribute string consensus_sequence;

  index< propagate=on, type=btree> on alignment_length;
  index< propagate=on, type=btree> on diameter;
  index< propagate=on, type=btree> on radius_of_gyration;
  index< propagate=on> on frequent_protein_name.name;
  index< propagate=on> on obtention.protein_family;
};

enum FamilyType {
  family = 0,
  domain = 1,
  repeat = 2,
  motif = 3

};

class PfamScore {
  attribute float gathering_method[2];
  attribute float lowest_sequence_score;
  attribute float lowest_domain_score;
  attribute float highest_sequence_score;
  attribute float highest_domain_score;
};

class PfamEntry extends ProteinFamily {
  attribute string source;
  attribute PfamEntry * seed;
  attribute string alignment_method;
  attribute string building_method;
  attribute FamilyType family_type;
  attribute PfamEntry * full;
  attribute PfamScore  ls;
  attribute PfamScore  fs;

  index< propagate=on> on source;
  index< propagate=on> on alignment_method;
  index< propagate=on> on building_method;
  index< propagate=on> on seed;
  index< propagate=on, type=btree> on ls.lowest_sequence_score;
  index< propagate=on, type=btree> on ls.lowest_domain_score;
  index< propagate=on, type=btree> on ls.highest_sequence_score;
  index< propagate=on, type=btree> on ls.highest_domain_score;
  index< propagate=on, type=btree> on fs.lowest_sequence_score;
  index< propagate=on, type=btree> on fs.lowest_domain_score;
  index< propagate=on, type=btree> on fs.highest_sequence_score;
  index< propagate=on, type=btree> on fs.highest_domain_score;
};

class RhMarker extends Marker {
  attribute set<RhAssay *> rh_assays;
  attribute set<ChrAssay *> chrom_assign;
};

class RhAssay extends Assay {
  attribute RhSTS * sts;
  attribute Panel * panel;
  attribute string score;
  attribute int32 number_of_trials;

  index< propagate=on, type=btree> on number_of_trials;
  index< propagate=on> on sts;
  index< propagate=on> on panel;
};

class RhSTS extends STS {
  attribute set<RhAssay *> rh_assays;
};

class ChrAssay extends Assay {
  attribute Chromosome * chromosome;
  attribute string comments;
  index< propagate=on> on chromosome;
};

class CloneLibrary extends BioLibrary {
  attribute ostring tissues[];
  attribute string vector;
};

class ExpressionKeyWord extends KeyWord {
  attribute set<SequenceCluster *> clusters;
  index< propagate=on> on clusters;
};

class SequenceCluster extends DbObject {
  attribute int32 nb_sequences;
  attribute Gene * gene;
  attribute set<CytoBand *> cytobands;
  attribute bag<CrossReference *> locus;
  attribute set<ExpressionKeyWord *> expression;
  attribute set<STS *> sts;
  attribute SequenceAlignment protein_similarity[];
  attribute set<SequenceObject *> sequences;
  attribute set<Chromosome *> chromosomes;

  index< propagate=on, type=btree> on nb_sequences;
  index< propagate=on> on gene;
  index< propagate=on> on chromosomes;
  index< propagate=on> on cytobands;
  index< propagate=on> on expression;
  index< propagate=on> on sts;
  index< propagate=on> on sequences;
  index< propagate=on> on locus;
};

enum MoleculeStrand {
  mol_unknown_strand = 0,
  mol_single = 1,
  mol_double = 2,
  mol_mixed = 3,
  mol_other_strand = 255
};

enum MoleculeInfo {
  mol_unknown_info = 0,
  mol_genomic = 1,
  mol_pre_RNA = 2,
  mol_mRNA = 3,
  mol_rRNA = 4,
  mol_tRNA = 5,
  mol_snRNA = 6,
  mol_scRNA = 7,
  mol_peptide = 8,
  mol_other_genetic = 9,
  mol_genomic_mRNA = 10,
  mol_cRNA = 11,
  mol_snoRNA = 12,
  mol_transcribed_RNA = 13,
  mol_other_info = 255
};

class GenbankEntry extends SequenceObject {
  attribute int32 gi;
  attribute MoleculeStrand strand;
  attribute MoleculeInfo molInfo;
  index< propagate=on> on gi;
};

class GenbankSequence extends GenbankEntry {
  attribute array<GenbankCDS *> cds;
};

class GenbankContig extends GenbankSequence {
  attribute SequenceAlignment sequences[];
};

class GenbankCDS extends GenbankEntry {
  attribute int32 frame;
  attribute int32 translation_table;
  attribute GenbankSequence * source;
  attribute SequenceLocation * location;
};

class TeraprotEntry extends SwissprotEntry {
  attribute int32 matid;
  attribute TeraProteome * proteome;
  attribute set<TeraprotCluster *> clusters;
  index< propagate=on> on description;
};

class TeraprotCluster extends BioObject {
  attribute set<TeraprotEntry *> proteins;
};

class TeraProteome {
  attribute string<128> name;
  attribute string file;
  attribute int32 matid;
  attribute set<TeraprotEntry *> proteins;
  index< propagate=on, type=btree> on name;
  index< propagate=on> on file;
};

class TeraprotOrganism extends Organism {
  attribute set<TeraProteome *> proteomes;
};

class TeraprotTaxon extends Taxon {
  attribute set<TeraProteome *> proteomes;
};
\end{verbatim}

\end{document}
